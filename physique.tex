\documentclass{article}
\usepackage[utf8]{inputenc}
\usepackage{amsmath}

\begin{document}
	\title{Révisions de Physique}
	\tableofcontents
\section{Propagation des Ondes}
	La propagation d'une onde est régit par la loi : $\frac{\partial^2s}{\partial t^2}-V^2\frac{\partial^2s}{\partial x^2}=0$
	Il ne faut pas oublier certaines relations de base : $p=mv=\frac{h}{\lambda}$ et
	$E=h\nu=\frac{hc}{\lambda}$
\section{Propagation des ondes mécaniques}
\subsection{Le long d'une corde}
Sur une corde avec une tension $T_0$ et une masse linéique $m_l$, la propagation des ondes :
$\frac{\partial^2u}{\partial x^2}=\frac{1}{V^2}\frac{\partial^2u}{\partial t^2}$
Avec : $V^2=\frac{T_0}{m_l}$\\
Les ondes se déplacent donc avec une vitesse de $V=\sqrt{\frac{T_0}{m_l}}$
\subsection{Dans un tuyau fermé remplit d'un fluide parfait}
Soit $p$ la surpression constaté en un endroit, la loi de la dynamique nous amène à :
$$ \left[\cfrac{\partial p}{\partial z}\right] = -\rho_0\left[\cfrac{\partial^2u}{\partial t^2}\right]$$
$\rho_0$ est la masse volumique du fluide. La surpression est la conséquence de la propagation d'une onde dans un tuyau à parois rigides.\\
Au sein des tuyaux, on a deux coefficients : $\chi$, le coefficient de \textbf{compressibilité} et $\kappa$, le module de \textbf{compression}. Ils sont reliés par la relation $\chi=\frac{1}{\kappa}$

\end{document}